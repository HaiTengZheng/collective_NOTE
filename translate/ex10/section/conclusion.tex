\section{结论}
为了解决严寒地区室内空气质量下降的问题,开发了新风与回风混合的空气源热泵新风机组,并在哈尔滨搭建了实验台进行系统测试。结论如下。

\begin{itemize}
	\item 通过新风与回风混合提高冷凝压力,使空气源热泵新风机组在严寒地区稳定运行,能效高。即使室外温度降至$\qty{20.9}{\degreeCelsius} $ ,热泵COP仍达到2.46,而电预热、热回收和空气源热泵相结合的新风系统 COP 仅为 1.44,而新风COP 仅 1.44。电预热、热回收和电加热相结合的系统仅为 1.21。随着室外环境温度的升高,机组的制热量和压缩机的功耗降低,每小时的 COP 随着室外温度的升高而增加。当室外温度为$\qty{4.9}{\degreeCelsius} $时,热泵COP为4.29。
	\item 当室外温度基本恒定且回风量较大时($\qty{1400}{\m^3/\hour} $、$\qty{1800}{\m^3/\hour} $),新风机组制冷剂参数波动不大,压缩机运行相对稳定,送风温度偏差为$\pm \qty{1}{\degreeCelsius} $,当回风量较小时$\qty{400}{\m^3/\hour} $,由于流经节流装置的液流不足,导致排出温度和排出压力波动较大,导致运行不稳定。压缩机的温度偏差为$\pm \qty{3}{\degreeCelsius} $。
	\item 机组的 COP 随着回风量的增大而增大。回风量$\qty{1800}{\m^3/\hour} $ 的 COP 比回风量$\qty{1400}{\m^3/\hour} $的 COP 高 0.15-0.23 。回风量的增加引起压缩机的吸气压力和排气压力
	\item 送风温度受机组出力影响。随着设定温度的升高,机组在满负荷运行的情况下很难满足新风加热的设定要求。热泵COP随着送风设定温度的变化呈现先增大后减小的趋势。本研究中获得的最佳设定温度为$\qty{24}{\degreeCelsius} $。
\end{itemize}
