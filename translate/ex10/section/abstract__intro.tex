\begin{abstract}
为保证建筑室内空气质量,针对严寒地区新风采暖问题,提出一种新风与回风混合的空气源热泵新风机组。新风与回风的混合过程在混合箱内实现,提高了冷凝器进风温度,创造了合适的冷凝压力,保证了机组节能稳定运行。为了探索机组的运行效果、供热性能和节能效果,在哈尔滨搭建了实验台,用于实验室供热。实验结果表面,当回风量为$\qty{1400}{\m^3/\hour} $时,在保证新风量$\qty{700}{\m^3/\hour} $、送风温度$\qty{24.0}{\degreeCelsius} $的条件下,新风机组能够稳定运行。室外温度为$\qty{4.9}{\degreeCelsius} $,热泵 COP  为 4.29。即使室外温度降至$\qty{20.9}{\degreeCelsius} $,热泵 COP 仍达到 2.46,而电预热、热回收和空气源热泵组合新风系统的 COP 仅为 1.44,而电预热、热回收和空气源热泵组合新风系统的 COP 仅为 1.44。电预热、热回收和电加热仅为 1.21。此外,热泵 COP 随着回风量的增加而增加。随着设定温度的升高,热泵 COP 先升高后降低。 

\noindent \textbf{关键字:} 新风机组,新风回风混合,空气原热泵,寒冷地区服务,COP
\end{abstract}

\section{简介}
社会经济的发展依赖于大量的能源。随着人口的增长、城市化进程的加快和建筑业的扩张,建筑能耗不断增加。在碳达峰和碳中和的背景下,减少建筑能耗是减少二氧化碳排放最经济有效的方法之一。减少建筑热损失最常见的方法是增加建筑物的气密性以减少空气渗透和增加建筑维护结构的隔热性能。然而,气密性高的建筑往往会导致室内空气质量较差,甚至出现病态建筑综合症。此外,如果室外空气直接通过开口或扩散器进入建筑物,居住者由于室外温度非常低,经常会感到有一股风的感觉。同于,由于雾霾等原因,冬季室外空气质量较差,阻碍了用户的自行开窗行为。独立新风系统的作用是提供新鲜空气并去除或稀释室内空气中的污染物。由于建筑物的使用寿命较长,可持续 50 年甚至更长,新风系统在节能方面采取回收建筑能耗的策略非常重要。据估计,热回收系统通常回收废气中约 60\% \textasciitilde 95\% 的热量,并显著提高建筑物能耗的能源效率。用于房间或建筑物新风系统的能量回收装置主要包括空气---空气热交换器、热管和旋转热轮。

空气---空气热交换器在能量再生中发挥着重要作用,利用建筑物废气中的能量来减少能源消耗。Seaara 等人设计了一种带有逆流装置的显热聚合物板式空对空换热器,用于新风预热,结果表明,换热器的效率随温度、相对湿度和新风流量的变化而变化。Vorayos 等人指出凹痕排列和凹痕间距对传热效率的影响。Al-Zubaydi 等人比较了热回收中两种不同板式换热器的性能,结果表面,凹坑式表面换热器的冷却能力比普通换热器提高了 50\% \textasciitilde 60\%。然而,当室外温度低于冰点,可能会结冰。使用热交换器,当水分进入时会发生冻结,从建筑物排出的空气在其表面凝结,交换器温度低于冰点。热交换器效率的提高会增加热交换器中的冷凝和结霜现象,这反过来又会降低热交换器的效率,并减少排气侧的压降。热交换器的效率,排气侧的压降也会增加。在新建筑中,加热板通常用于将进气预热到 0 °C 以上,以解决结霜问题。但这会消耗额外的能源。

热管由充满工作流体的密封管道组成,通过冷凝和蒸发传递热量。通过冷凝和蒸发来传递热量。Yang 等人利用热管热交换器回收汽车尾气中的废热,以加热车厢内的冷空气。实验结果表明该方法是可行的。Xue 等人使用热管插入式板式空气---空气热交换器回收废气中的能量,实验结果表明该方法是可行的。实验结果表明,在冬季条件下,最高温度效率达到 62\%。Diao 等人设计了一种小型扁平热管热回收装置回收装置处理新风,在实验条件下最高热回收率可达 78\%。Zhu 等人设计并建造了一个利用微型热管阵列进行热回收的空气-空气热交换器。结果表明,热管热交换器效率高,节能效果显著。Zhou 等人提出了一种泵驱动循环热管系统,用于回收废气中的冷/热能量,对新鲜空气进行连续预冷/预热。结果表明,R32 作为工作流体的整体性能优于 R22 和 R152a。虽然热管具有无活动部件、冷热流体完全分离和可靠性高等优点,但其成本较高,在供暖、通风和空调领域的市场渗透率较低。

旋转轮在两股气流之间旋转,并在它们之间交换热量。Ghodsipour 等人建立了旋转热轮的数学模型,并对其进行了求解。并得出结论:影响旋转热轮效率的主要参数是旋转速度、气流速度和热量交换速度。O'Connor 等人提出了旋转热轮热回收集成系统。并表明,尽管热力转轮堵塞,但通风率高于推荐值,同时排气气流中的热量被收集并转移到进气气流中,温度也会随之升高。进气流,温度提高了$\qty{2}{\degreeCelsius} $。Herath 等人研究了回转热轮回收空调机组废气的可用能量。结果表明,随着新风温度和相对湿度的增加,转轮的节能率也随之增加。Calautit等人开发了一种旋转热回收装置,适用于与安装在屋顶上的多向捕风器系统结合使用。除了充分通风外,室内空气温度也提高了 3.5\%。温度也提高了$\qty{3.7}{\degreeCelsius} $。热轮的独特优势在于热轮的独特优势在于它可以回收显热和潜热。但它有活动部件,增加了维护成本、此外,还应考虑结霜问题。

在严寒地区,采暖季室外温度极低,如哈尔滨的极端最低温度为$\qty{37.7}{\degreeCelsius}$。而且降雪频率高,因此新风系统需要有较大的加热能力。传统的热回收装置存在结霜问题,需要通过较大的预热温差来避免。考虑到能源的便利性和 考虑到能源的便利性和系统的简易性,这部分预热负荷通常由电加热承担。这与低碳节能的理念背道而驰。在碳中和的目标下、普及空气源热泵等可再生能源技术是建筑节能的重要途径。

Wei 等人对严寒条件下的准两级压缩空气源热泵系统进行了实验研究。结果表明,该系统可在整个采暖季平稳运行。整个采暖季。与集中供暖和电锅炉相比,一次能源效率分别提高了18.67\% 和 140.54\%,而污染物排放则分别减少了 15.41\% 和 58.27\%。Wei 等人对变制冷剂流量供热系统进行了实验研究。研究了严寒地区的喷气式变制冷剂流量热泵系统。注汽式变制冷剂流量热泵系统进行了实验研究。结果表明哈尔滨的季节能效系数为 2.40,室内温度基本能满足供暖需求。与此同时,对空气源热泵蒸发器的除霜问题也进行了大量研究。冬季空气源热泵蒸发器的除霜问题进行了大量研究。赵等人提出了一种可快速加热和除霜的储能式空气源热泵系统,并对其进行了研究。
并进行了实验研究。结果表明,除霜时间比传统除霜缩短了 68\%,除霜能耗降低了 51.5\%,化霜能耗降低了 51.4\%。Qu 等人提出了一种基于热能储存的逆循环除霜方法,并进行了实验研究。结果表明,与标准的热气旁路除霜方法相比,除霜时间缩短了 71.4\%--80.5\%,除霜能耗降低了 65.1\%--85.2\%。Zheng 等人提出了一种温度---湿度---图像方法,并进行了实验研究。结果表明, 该方法直接影响霜层的形成,从而制定更合理的除霜策略。空气源热泵技术因其高效节能和环保而被广泛应用于建筑物的空间供暖。

 但是,当空气源热泵直接用于加热新风时,由于室外蒸发器侧和室内冷凝器侧的温度都是室外温度,冷凝压力和蒸发压力的差值会过小,节流装置的液流量不足,为了解决此类问题,有时会通过恶化冷凝器的换热来提高冷凝压力。一方面降低了系统的能效,同时冷凝器的风量不足,无法调节,新风量也无法保证。针对上述问题,本文提出了一种新风与回风混合的新型空气源热泵新风机组。本文的创新点如下。(1) 将室内回风与室外新风混合,提高冷凝器入口温度,形成合适的冷凝压力,再利用空气源热泵的室内热交换器(冷凝器)对混合空气进行加热,从而实现新风量的可调可控。(2) 新风与回风的混合过程在混合箱内实现,无运动部件,组件数量少,最大程度地避免了换热器结霜的问题。(3) 与普通空气源热泵相比,提高冷凝温度能充分体现空气源热泵的节能特性。该系统为严寒地区的新风加热提供了新思路,改善了室内空气质量,解决了新风加热过程中能源使用不合理等问题。本文研究了严寒地区新风与回风混合的空气源热泵新风机组的应用特点,拓展了其应用领域,并对其在严寒地区的应用前景进行了展望。 本文研究了新风与回风混合的空气源热泵新风机组在严寒地区的应用特性,以拓展其应用领域,并建立了性能测试台。性能试验台。本文第二部分介绍了 实验原理、空气处理过程和数据收集与处理。和数据处理。第三部分主要分析了该装置在特定运行条件下的性能 设备在特定运行条件下的性能。回风量和设定温度变化时的性能。
