\documentclass{ctexart}
% \documentclass{amsart}
 


%%%%%%%%%%%%%%%%%%%%%%%%%%%%%%%%%%%%%%%%%%%%%%%%%%%%%%%%%%%%%%%%%%%%%%%%%%%%%%%
% language support
\usepackage{xeCJK}

% indentfirst
%% 添加首行缩进,两个字符,用于中文文档
%% 用于英文文档时注释以下两行
\usepackage{indentfirst}
\setlength{\parindent}{2em}

%xcolor
\usepackage[dvipsnames]{xcolor}

% math
\usepackage{amsfonts,amsmath,amssymb,amsthm}
\usepackage{bm}
\usepackage{mathtools}
%% define new
\newtheorem{definition}{Definition}[section]
\newtheorem{axiom}{Axiom}[section]
\newtheorem{theorem}{Theorem}[section]
\newtheorem{corollary}{Corollary}[theorem]
\newtheorem{lemma}[theorem]{Lemma}
\newtheorem*{remark}{Remark}
\newtheorem*{exercise}{Exercise}

% hyperref
\usepackage{hyperref}
\hypersetup{
	colorlinks=true,
	linkcolor=blue,
	pdftitle={ex9},
}

% acronym
% \usepackage[acronym]{glossaries-extra}
% \setabbreviationstyle[acronym]{long-short}
% \makeglossaries
%% e.g. \newacronym{bas}{BAS}{Building Automation System}

% tcolorbox
\usepackage[most]{tcolorbox}

% geometry
%% it's useful to adjust page margins and line spacing.
% \usepackage[left=2cm, right=2cm, top=1.54cm, bottom=1cm]{geometry}
\usepackage[margin=2.5cm]{geometry}

% fancyhdr
%% fancyhdr 需要放在 geometry 后面
\usepackage{fancyhdr}
\pagestyle{fancy}
\fancyhead[L]{}
\fancyhead[C]{宁波工程学院毕业设计(论文)---外文翻译}
\fancyhead[R]{}
\fancyfoot[L]{}
\fancyfoot[C]{\thepage}
\fancyfoot[R]{}



% setspace
%% a simple way of changing line spacing
%% e.g. for 1.5x line spacing
% \usepackage[onehalfspacing]{setspace}
%% [doublespacing] for 2x line spacing

% siunitx
\usepackage{siunitx}

% graphicx
%% include the graphics
\usepackage{graphicx}
\usepackage{float}
\usepackage{subfigure}
%% e.g. scale the image to 80% of the text width
% \includegraphics[width=0.8\textwidth]{figure.eps}
%% e.g. fix the hight and width of a figure
%% [width=2cm, height=5cm]
%% e.g. rotate 90 degrees
%% [angle=90]
%% set the default path
% \graphicspath{{./figures}{./additionalFigures}}

% biblatex
% \usepackage{biblatex}

% lineno
%% 每行数字序列
% \usepackage{lineno}
% \linenumbers

% mhchem
\usepackage[version=4]{mhchem}

\usepackage{booktabs}
\usepackage{multirow}

% define symbol
\newcommand{\bsl}{\texttt{\symbol{92}}}

