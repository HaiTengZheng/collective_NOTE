\begin{abstract}
作为一种清洁、可持续电气化供暖技术,空气源热泵 (ASHP) 已在中国家庭中得到广泛应用。然而,常用的水循环系统存在效率低、结构复杂、易冻裂、电能消耗高等问题。本文提出了一种新型多组热管散热空气源热泵 (ASHPMP)。在 ASHPMP 系统中,冷凝器和热管耦合在一起,形成一种新型的热辐射终端,被命名为热管散热器,由压缩机驱动的多个终端为多个房间提供热量。开发 ASHPMP 系统的实验装置,并在住宅楼中进行了实验,热泵使用 R410A,热管使用 R134a。分析结霜与除霜条件下温度、热泵和热管压力对 ASHPMP 供热性能的影响。结果表明,新系统可在室外温度为$\qty{-12.7}{\degreeCelsius} $至$\qty{6.5}{\degreeCelsius} $的条件下稳定运行. 当室内温度设定为$\qty{20}{\degreeCelsius} $, 室外温度为$\qty{-12.7}{\degreeCelsius} $时, 制热性能系数(制热 COP)可达 3.15, 室外温度为$\qty{6.5}{\degreeCelsius} $时, 制热性能系数为 6.73.

\noindent \textbf{关键字:} 空气源热泵; 热管散热器; 供热性能; 应用研究
\end{abstract}

\section{导言}
研究表明,燃烧导致的的碳排放呈上升趋势。随着经济的快速发展,煤碳消耗量不断增加,环境污染日益严重。在中国京津冀地区, 2014 年燃煤造成的 PM2.5 浓度占年均值的 28.5\%。最近几年,中国大多数城市的空气质量已引起人们的关注,持续减少燃煤消耗是唯一的办法,尤其是在农村地区冬季取暖的建筑能耗居高不下的情况下。为减少 \ce{CO2} 排放和 PM2.5,中国实施了一系列政策和措施,如“煤改电”计划,重点是用电采暖设备替代化石燃料。

热泵是一种具有不同用途的高效换热设备,可利用各种能源工作。文献提出了一种利用浴室排水余热的新型热泵系统。文献将实验室的热泵装置改装成了冰箱,并使用对使用 R407C 制冷剂的新系统性能进行了研究。近年来,空气源热泵 (ASHP) 已经在中国北方地区,尤其是家电领域发挥着重要的作用。对于 ASHP 供热系统,散热器和空调 (AC) 是常用的供热终端,在中国的“京津冀”地区,90\% 以上的居民选择了热辐射终端。水流经室外机的冷凝器,从约$\qty{40}{\degreeCelsius}$加热到$\qty{60}{\degreeCelsius}$,然后在水泵的驱动下流入散热器,加热室内空气。

但也发现了一些问题,如效率低、结构复杂、更容易冻结和爆裂以及电能消耗高。为了避免二次换热,简化系统结构,直接将冷凝末端与 AHSP 相结合,许多研究都guan关注空调的室内热舒适性。事实上,空调的热不适感稍强,如强烈的穿堂风和室内噪音。作为另一种直接冷凝式采暖终端,AHSP 系统的制冷剂直接加热地板,许多研究都集中在直接地板辐射采暖的传热过程上。

作为一种高效的热交换器,热管被广泛应用于核反应堆冷却、电子设备 CPU 冷却,建筑物冷却、太阳能利用等。利用热管集热器/蒸发器作为蒸汽压缩热泵的另一种热源,构建了实验装置,结果表明机组的平均热容量和制热 COP 可分别提高 27.61\% 和 27.85\%。在冷凝器中插入热管以改善空调冷凝测的散热性能,实验结果表明能效比比传统空调高出 3.11\%。热管也可用于地源热泵系统,热管是收集低品味地热的关键部件。

在我们之前的研究中,提出了以热管作为散热器的 ASHP 系统,压缩机驱动热管向室内释放热量。实验研究了热管内部的传热过程、工作流体类型和充注量对系统性能的影响和系统工作性能随工况的变化等。文献验证了热管在空气源热泵中应用的可行性:
\begin{itemize}
    \item 热管表面的温度均匀性非常好,表面平均温度与热泵leng冷凝温度之间的差异非常校,仅为$\qty{1.5}{\degreeCelsius}$。
    \item 在室外温度为 -$\qty{15}{\degreeCelsius}$,室外温度为$\qty{22}{\degreeCelsius}$的情况下,新型 ASHP 系统的制热 COP 可达 2.55。
\end{itemize}

然而,这些研究只提出了但热管散热器系统的模式和原理,系统性能也是在实验室中研究的。对于谁环热泵系统来说,热水是通过水泵的驱动输送到多个房间的采暖末端。为了满足实际建筑的供暖需求,还需要将热泵系统做成多端型。当末端数量增加时,热管温度均匀性、系统稳定性、能效等方面的考虑就显得尤为重要。特别是在除霜工况下运行时,除霜效果、热管表面温度的下降值对其实际应用至关重要。

本文提出了一种带有多组热管散热器 (ASHPMP) 的空气源热泵系统。为了便于安装和将室内和室外机连接,多根热管的末端被串联起来。开发了一种实验方法,并在环境控制室进行了测试。压力、温度和其他参数
在实际供暖建筑和实际气象条件下对热泵和热管的压力、温度和其他参数进行了实验研究。在实际供暖建筑和实际气象条件下,对热泵和热管的压力、温度和参数进行了实验研究。还探讨了霜冻/结霜条件下系统供热性能的变化。还探讨了霜冻/结霜条件下系统供热性能的变化。

