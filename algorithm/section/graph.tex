\section{Graph}
\subsection{Definition}
\begin{definition}[graph]
	A graph G = (V, E) consists of a set of vertices, V, and a set of edges, E.
\end{definition}

Each edge is a pare $(v, w)$, where $(v, w) \in V$. Edge are somtimes referred
to as \emph{arcs}. 

\begin{definition}[directed]
	Pair is ordered.
\end{definition}

Directed graphs are somtimes referred to as \emph{digraphs}.

Vertex $w$ is adjacent to $v$ if and only if $(v, w) \in E$.

In an undirected graph with edge $(v, w)$, and hence $(w, v)$, $w$ is adjacent
to $v$ and $v$ is adjacent to $w$.

Somtimes an edge has a third component, known as either a \emph{weight} and 
a \emph{cost}.

\begin{definition}[path]
	A path in a graph is a sequence of vertices $w_1, w_2, w_3, \dots , w_N$
	such that $(w_i, w_{i+1}) \in E$ for $i \leq i < N$.
\end{definition}

A \emph{length} of such a path is the number of edges on the path, which is 
equal to $N - 1$. We allow a path from a vertex to itself; if this path contains
no edges, then the path length is 0. The $(v, v)$ is somtimes referred to as a
\emph{loop}.

\begin{definition}[simple path]
	A simple path is a path such that all vertices are distinct, exepect that 
	the first and last could be the same.
\end{definition}

\begin{definition}[cycle]
	A cycle in a directed graph is a path of length at least 1 such that 
	$w_1 = w_N$.
\end{definition}

A directed graph is \emph{acyclic} if it has no cycles. A directed acyclic
graph is somtimes referred to by its abbreviation, \emph{DAG}.
