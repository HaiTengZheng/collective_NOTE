\section{Macro}
Rust macros don't capture variables from their environment.

Rule of thumb: macros can be used in situations where functions fail to provide
the desired solution, where you have code that is quite repetitive, or in cases
where you need to inspect the structure of your types and generate code at
compiler time.

Rust do their magic of code generation before the program compiles to a binary
object file. They take input, known as \emph{token trees}, and are expanded at
the end of the second pass of parsing during \gls{ast} construction.

Rust macros are parsed at the end of the second phrase of the \gls{ast} 
construction, which is a phase where name resolution happens.

\subsection{Types of macros}
\paragraph{Declarative macros}
The simplest form, created using \verb"macro_rules!".
\begin{lstlisting}[language=rust, style=boxed]
use std::io;
macro_rules! scanline {
	($x:expr) => ({
		io.stdin().read_line(&mut $x).unwrap();
	});
	() => ({
		let mut s = String::new();
		io.stdin().read_line(&mut s).unwrap();
		s
	});
}
/* `scanline` is the macro name */
\end{lstlisting}

Every matching rule consists of these parts.
\begin{itemize}
	\item The \emph{pattern matcher}, that is, the \verb!($x:expr)!
		  part. 
	\item A \verb!=>!.
	\item The \emph{code generation block}, which can be delimited
		  either with \verb!()!, \verb!{}!, or even \verb![]!.
\end{itemize}

A matching rule has to end with a semicolon when there is more than one rule to
match. \verb!$x! is a token tree, \verb!expr! is token tree type which means
that this macro can only accept things that are expressions.

The token tree types list below:
\begin{itemize}
	\item \verb!block!
	\item \verb!expr!
	\item \verb!ident!
	\item \verb!item!
	\item \verb!meta!
	\item \verb!pat!
	\item \verb!path!
	\item \verb!stmt!
	\item \verb!tt!
	\item \verb!ty!
	\item \verb!vis!
	\item \verb!lifetime!
	\item \verb!literal!
\end{itemize}

\paragraph{Procedural macros}
Advanced form, have complete control over the manipulation and generation of 
code. These macros don't come with any \gls{dsl} support and are procedural in 
the sense that you have to write how you want the code to be generated or
transformed for a given token tree input.


