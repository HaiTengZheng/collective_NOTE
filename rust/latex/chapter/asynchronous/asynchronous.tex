\chapter{Asynchronous Programming}

\section{Fundamental Concepts}
\begin{definition}[Asynchronous Programming]
	A concurrent programming model. 
\end{definition}

\emph{Async} improve the system performance by nont blcoking the threads.

\begin{tcolorbox}
	\begin{quotation}
		It lets you run a large number of concurrent tasks on a small number of OS threads, 
		while preserving much of the look and feel of ordinary synchronous programming, 
		through the \emph{async/await} syntax.
	\end{quotation}
\end{tcolorbox}
\begin{tcolorbox}
	\begin{quotation}
		相比与传统的同步编程,异步编程可以更好地处理 IO 密集型任务和并发请求,提
		高系统的吞吐量核性能。
	\end{quotation}
\end{tcolorbox}

\section{Other Concurrency Models}
\begin{itemize}
	\item OS threads
	\item Event-driven programming
	\item Coroutines
	\item The actor model
\end{itemize}

\section{Future}
\emph{Future} is an asynchronous computation that will be completed in the future. 
\begin{lstlisting}[language=rust, style=boxed]
pub trait Future {
	type Output;

	// Require method
	fn poll(self: Pin<&mut Self>, cx: &mut Context<'_>) -> Poll<Self::Output>;
}

pub enum Poll<T> {
    Ready(T),
    Pending,
}
\end{lstlisting}

Async in Rust uses a \emph{poll-based approach} in which an asynchronous task will
have three phases.
\begin{enumerate}
	\item \emph{The poll phase:}
		  We often refer to the part of the runtime that polls a future as an \emph{executor}.
	\item \emph{The wait phase:}
		  An event source, most often referred to as a \emph{reactor}.
	\item \emph{The wake phase:}
		  The event happens, and the future is woken up.
\end{enumerate}

\subsection{Leaf futures}
\begin{definition}[Leaf futures]
	Runtime create it, which represent a resource such as a socket.
\end{definition}

\subsection{Non-Leaf futures}
\begin{definition}[Non-Leaf futures]
	The kind of futures we as users of runtime write ourselves using the \emph{async}
	keyword to create a task that can be run on the executor.
\end{definition}

\subsection{A mental model of an async runtime}
Rust doesn't come with a runtime for handling concurrency.
\begin{itemize}
	\item  \emph{Reactor} (responsible for notifyling about I/O events)
	\item  \emph{Executor} (scheduler)
	\item  \emph{Future} (a task that cans stop and resume at specific points)
\end{itemize}

\section{Resource Collection}
\begin{itemize}
	\item \href{https://rust-lang.github.io/async-book/01_getting_started/01_chapter.html}
			   {Asynchronous Programming in Rust}
	\item \href{https://course.rs/advance/async/intro.html}
			   {Rust Course}
	\item \href{https://github.com/smallnest/concurrency-programming-via-rust}
			   {concurrency-programming-via-rust}
	\item \href{https://github.com/PacktPublishing/Asynchronous-Programming-in-Rust}
			   {Asynchronous Programming in Rust (By Carl Fredrik Samson)} 
	\item \href{https://github.com/PacktPublishing/Hands-On-Concurrency-with-Rust}
			   {Hands-On-Concurrency-with-Rust}
\end{itemize}

